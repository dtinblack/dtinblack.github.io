\documentclass{article}[12pt]
\usepackage{bookman}
\usepackage{amsmath}                      
\usepackage[parfill]{parskip}
\usepackage{hyperref}
\usepackage{xcolor}
\usepackage{graphicx}

\title{Turing's Argument For Unsolvable Problems}
\author{David Black}

\begin{document}
\maketitle

\section*{Introduction}

\noindent
At the heart of Turing's article for the \emph{Science News}  about solvable and unsolvable problems there is argument that he uses to show that there is no systematic way to decide whether a particular type of puzzle, a substitution puzzle, is decidable\cite{Turing1}.  But in the middle of the argument he makes a remark that is not clear for me. 

\section*{The Argument} 

\noindent

Let $P(R,S)$ stand for ‘the puzzle whose rules are described by the row of symbols $R$ and whose starting position is described by $S$’. 

\noindent

(i) Owing to the special form that is used to describe the rules i.e. reducing them to a string of symbols \cite{Turing1}, then we can consider $P(R,R)$ where the ‘rules’ also serve as starting position. 

\noindent

(ii)  A puzzle is said to have ‘come out’ if one reaches either the position B or the position W, and the rules do not permit any further moves. 

\noindent

(iii) The rules $R$ of the puzzles can be classified:

\noindent

Class I is to consist of rules $R$ such that $P(R,R)$ 'comes out' with the end result $W$.

\noindent

Class II is to include all other cases, i.e. either $P(R,R)$ does not come out, or comes out with the end result $B$, or else $R$ does not represent a puzzle with unambiguous moves. 

\noindent

(iv) Assume that there that there is a systematic procedure, $K$ that can represent the classification described in (iii), with the following properties:

\noindent

$K$ has unambiguous moves.

\noindent

$P(K,R)$ always comes out whatever $R$.  

\noindent

If $R$ is in Class I, then $P(K,R)$ has end result $B$. 

\noindent

If $R$ is in Class II, then $P(K,R)$ has end result $W$.

\noindent

It is important to note that there is no reason for selecting $B$ or $W$ for the result of $K$. 

\noindent

(vi)  The question is - which Class does $K$ belongs to? 

\noindent

The properties of $K$ tell us that we must get end result B if K is in Class I and W if it is in Class II, whereas the definitions of the classes (iii) tell us that the end results must be the other way round.  Therefore the assumption (iv) that there was a systematic procedure for telling whether puzzles come out produces a contradiction and therefore doesn't exist.

\section*{Discussion}

\noindent

The problem for me is that at step (iv).  Turing says - page 592 - inthe attached paper - "The opposite choice would be equally possible, and would hold for a slightly different set of rules $K^\prime$, which however we do not choose to favour with our attention." Where I am struggling is that if the properties of  $K^\prime$ were W in Class I and B in Class II then there is no contradiction - how is the remark of any help?
\noindent

I have come up with the following thoughts:

\noindent

(a) by stage (iv) a contradiction has been shown and therefore no more needs to be said.

\noindent

(b) If the classification (iii) was reversed then we would get a contradiction but I feel that this is not what Turing is saying.

\noindent

(c) I tried different ideas about the rules for $K^\prime$ e.g. add rules to $K^\prime$ could include:

\noindent
if the end result is W then switch the end result to B, and 

\noindent

if the end result is B then switch the end result to W.

\noindent

but this looks very clumsy and probably isn't what he meant either.

\noindent

I'm clearly missing something - what is it?

\noindent

\section*{Discussion on Stack Exchange}

\noindent

A ) Obviously I cannot be sure what Turing was thinking while writing the basis for the paper you're citing. It is true that $K^\prime$ 
doesn't matter for the specific reasoning in the paper, because it isn't contradictory.
\noindent

B ) But, if you do consider $K^\prime$ , what it is is a hypothetical puzzle that accurately reports the class of every other puzzle, using $B$
to mean Class II since that is the class that contains the puzzle that come out $B$. While this is not directly contradictory, it is possible to modify $K^\prime$  to obtain a puzzle that satisfies all the premises for $K$, by swapping the results as you mentioned. So, it turns out that $K^\prime$ 
cannot exist either.

\noindent

C ) In fact, a typical proof for the undecidability of the halting problem on Turing machines actually builds something like $K^\prime$ 
as an intermediate step. You start with a puzzle $H$ such that $P( H, R*S)$ supposedly reports if/how $P(R,S)$
comes out, similarly to the two classes (where $R*S$ pairs up a puzzle description with a starting position in a way that $H$
can handle). Then you modify it to build $K^\prime$  such that $P(K^\prime, R )=P (H, R*R )$. Then you invert the result to get the puzzle $K$.

\noindent

D )  So, often $K^\prime$  is involved in the actual proof. In this article Turing is only describing the final list of contradictory properties for the rules $K$
, and you are just expected to accept that $K$'s existence is a consequence of the existence of something like $H$ (or even vaguer criteria). But full proofs show how to build $K$ from $H$ in detail, and $K^\prime$  will often appear.

\noindent

\textbf{Reply:}

Thanks for your reply to my question - it has helped to clarify my own understanding of his article. 

The article that he wrote was for Science News which was aimed at an audience with a general interest in science and therefore showing that $K$ produced a contradiction was enough to make his point.

I followed up on the point that you made about the proof of the The Halting Problem and how it builds something like $K^\prime$ and $H$. There is thousands of descriptions of The Halting Problem on the web but the best one that I found is in Marvin Minsky's book "Computation: Finite and Infinite Machines" pp146 - 149 which I feel best describes the point that you made.  Following the the form of his argument: because $K$ produces a contradiction, therefore it doesn't exists and neither does "$K^\prime$ , and subsequently $H$. 



\noindent - other thoughts - decision machines would better suit a puzzle - does the substitution idea come from Emil Post?

\noindent

Question: Did Turing use some of mPost’s ideas about symbol manipulation to argue about the some puzzles being unsolvabileity of some types of puzzle ( and not what I though he was indication e.g. a Turing Machine )?

Aware of Post’s work - his annotated copy sold by Christie’s - https://www.christies.com/en/lot/lot-5791610 - therefore he was aware.  see 'Alan Turing his work and impact' p344 where there is a clear connection to Post's work

Check - https://projecteuclid.org/journals/journal-of-the-mathematical-society-of-japan/volume-13/issue-1/Elementary-formal-systems/10.2969/jmsj/01310038.pdf


\section*{Conclusion}


\appendix
\section{ Russell's Paradox} \label{A:Russell's Paradox}

The comprehensive class we are considering, which is to embrace everything, must embrace itself as one of its members. In other words, if there is such a thing as “everything,” then, “everything” is something, and is a member of the class “everything.” But normally a class is not a member of itself. Mankind, for example, is not a man. Form now the assemblage of all classes which are not members of themselves. This is a class: is it a member of itself or not? If it is, it is one of those classes that are not members of themselves, i.e., it is not a member of itself. If it is not, it is not one of those classes that are not members of themselves, i.e. it is a member of itself. Thus of the two hypotheses – that it is, and that it is not, a member of itself – each implies its contradictory. This is a contradiction. \cite{Russell1}

\noindent
Let $R$ be the class of all classes that are not members of themselves. Is it a member of itself or not?

\noindent

Suppose that it is. Then $R$ must meet the ( necessary ) condition for belong to $R$: that it is not a member of itself. So if it is a member of itself, it is not a member of itself. In other words: If $R \in R$ then $R \notin R$. 

\noindent 

Suppose that it is not. Then, being a nonself-membered Class, it meets the ( sufficient ) condition for belonging to $R$: that it is not a member of itself. So if it is not a member of itself, it belongs to $R$ and so is a member of itself. In other words:  If $R \notin R$ then $R \in R$.  

\noindent
Therefore  $R$ is a member of itself if and only if it is not a member of itself i.e. $R \in R \iff  R \notin R$ which is a contradiction.

\section{Barber's Paradox}

In a village there is a man who is the barber. This barber shares all and only those men in the village who do not shave themselves. Does the barber shave himself?

\noindent
If the  \emph{barbers shaves himself}, then he \emph{does not shave himself} because the barber is the only one who shaves the men who do not shave themselves.

\noindent
If the \emph{barber doesn't shave himself} then he \emph{shave by himself} because he shaves the men who do not shave themselves.

\noindent

Therefore a contradiction. \emph{ - need of mention Quine's The Ways of Paradox to classify this the type of paradox }

\section{The Halting Problem\cite{Minsky1}}

\noindent

Let us suppose, \emph{as a hypothesis to be proved self-contradictory}, that we have a machine $D$ which will decide whether or not any Turing machine computation will ever halt, given descriptions $d_t$ of machine $T$ and its tape $t$. Then $D$ has the form: 

\begin{equation*}                                       % adding * stops auto numbering
   D(d_t, t ) \ \text{outputs} \ \ \begin{cases}
    Yes \ ( \ then \ Halts \ ) \ \  \text{if $T$ eventually halts, given $t$} \\
    No \  ( \ then \ Halts \ ) \ \  \text{ if $T$ never halts, given $t$}.
  \end{cases}
\end{equation*}

\noindent 

Suppose we construct another machine $E$ that copies $dt_t$ and then runs $D$:

\begin{equation*}                                       % adding * stops auto numbering
   E(d_t) \ \text{outputs} \ \ \begin{cases}
    Yes \ ( \ then \ Halts \ )\ \ \  \text{if $T$ eventually halts, given $d_t$} \\
     No \ ( \ then \ Halts \ ) \ \ \  \text{ if $T$ never halts, $d_t$}.
  \end{cases}
\end{equation*}

\noindent 

Create new machine $E^*$ by adding two new states to prevent $E*$ from ever halting if $E$ takes the \emph{Yes} exit. $E^*$ has the form:
\noindent 

\begin{equation*}                                       % adding * stops auto numbering
   E^*(d_t) \ \text{outputs} \ \ \begin{cases}
    Yes \ ( \ then \ loops \ forever \ ) \ \ \  \text{if $T$ eventually halts, given $d_t$} \\
     No  \ (  \ then \ Halts  \ )  \ \ \ \ \ \ \ \ \  \  \  \ \ \  \text{ if $T$ never halts, $d_t$}.
  \end{cases}
\end{equation*}

\noindent 

What happens if $E^*$ is applied to $d_E*$ i.e. $E^*(d_E*)$?

\noindent

\emph{It halts if $E^*$ applied to $d_t$ does not halt, and vice versa}. This cannot be, so we must conclude that such a machine as $E^*$, and hence $E$, and hence $D$, could not exists in the first place!

\noindent

( Case A ) \emph{$E^*$ eventually halts}

\noindent

That is to say the $E^*$ halts given  $d_E*$. The $E$ takes its "Yes", given $d_E*$. But this means that when $E$ finished its computation on $d_E*$, it eventually enters its upper terminal state. Since $E^*$ is the same as $E$ up to this point, then $E^*$, given $d_E*$ eventually reaches \emph{its} uppper exit state and enters a never ending loop. Therefore  $E^*$ \emph{does not} halt given,  $d_E*$.


\noindent

( Case B ) \emph{$E^*$ never halts, given $d_E*$}

\noindent

But in a similar argument then $E$, given $d_E*$, takes its lower exit, and so $E^*$ ( which is identical to $E$ up to this point ) takes its lower exit and \emph{does} halt.


\noindent

Since there are no other possibilities, the existence of $E^*$ itself is a contradiction, hence that of $E$, hence the existence of $D$.


% bibliography 

\begin{thebibliography}{10}

\bibitem{Turing1}
Copeland, J, \emph{The Essential Turing}, Oxford University Press, 2004, pp 576-595, \href{https://www.ivanociardelli.altervista.org/wp-content/uploads/2018/04/Solvable-and-unsolvable-problems.pdf}{ \emph{Solvable and Unsolvable Problems}},  
\bibitem{Russell1}
Russell, B, \emph{Introduction to Mathematical Philosophy}, London: George Allen and Unwin Ltd, and New York: The Macmillan Co., 1910 p136.
\bibitem{Quine1} Quine, W. V., \emph{Ways of Paradox and Other Essays}, Harvard University Press, 1976, pp1-18,\href{http://www.thatmarcusfamily.org/philosophy/Course_Websites/Readings/Quine%20-%20Ways%20of%20Paradox.pdf}{ \emph{The Ways of Paradox}}
\bibitem{Minsky1} Minsky, L. M,  \href{https://archive.org/details/computationfinit0000mins/page/n5/mode/2up}{ \emph{Computation: Finite and Infinite Machines }} pp146 - 149

\bibitem{StackExchange1} Computer Science, \href{https://cs.stackexchange.com/questions/169803/turings-argument-for-unsolvable-problems}{\emph{Turing's Argument For Unsolvable Problems}}item

\bibitem{Burkholder1} Burkholder, L, \href{https://dl.acm.org/doi/pdf/10.1145/24658.24665}{The Halting Problem}

\end{thebibliography}

\end{document}